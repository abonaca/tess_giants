\documentclass[modern]{aastex63}

% typography
\usepackage[T1]{fontenc}
\setlength{\parindent}{1.\baselineskip}
\newcommand{\acronym}[1]{{\small{#1}}}
\newcommand{\package}[1]{\textsl{#1}}
\newcommand{\changes}[1]{{\textbf{#1}}}

% math
\newcommand{\peak}{\mathrm{peak}}

% aastex parameters
% \received{not yet; THIS IS A DRAFT}
%\revised{not yet}
%\accepted{not yet}
% % Adds "Submitted to " the argument.
% \submitjournal{ApJ}
\shorttitle{}
\shortauthors{bonaca \& hogg}

%@arxiver{}
\usepackage{amsmath}

\begin{document}\sloppy\sloppypar\raggedbottom\frenchspacing % trust me

\title{Likelihood for the forest of asteroseismic modes}

\correspondingauthor{Ana~Bonaca}
\email{ana.bonaca@cfa.harvard.edu}

\author[0000-0002-7846-9787]{Ana~Bonaca}
\affil{Center for Astrophysics | Harvard \& Smithsonian, 60 Garden Street, Cambridge, MA 02138, USA}

\author[0000-0003-2866-9403]{David~W.~Hogg}
\affiliation{Center for Cosmology and Particle Physics, Department of Physics, New York University}
\affiliation{Center for Data Science, New York University}
\affiliation{Max-Planck-Institut f\"ur Astronomie, Heidelberg}
\affiliation{Center for Computational Astrophysics, Flatiron Institute, 162 Fifth Avenue, NY 10010, USA}

\begin{abstract}\noindent % trust me
\end{abstract}

\section{Introduction}
\label{sec:intro}

\section{Likelihood}
\label{sec:lhood}

There are many ways to construct a likelihood function for an
asteroseismic-like frequency comb that generates a light curve.
We will build up in stages of increasing sophistication.
The simplest idea is that the light curve is generated by a set of
$K$ equally spaced frequencies.

Presume that there are $N$ data points (normalized fluxe measurements)
$y_n$, normalized to have a mean near unity.
The data points are taken at (precisely known) times $t_n$.
Each of these flux measurements has some kind of uncertainty estimate
$\sigma_n$ associated with it and inverse variance $\sigma_n^{-2}$.
The simplest $K$-frequency model is that these flux measurements are
generated by a set of $K$ frequencies $\nu_k$ (where $K$ is odd) as follows:
\begin{align}
  y_n &= \mu_n(t_n) + \mbox{noise}
  \\
  \mu_n(t_n) &= a_0 + \sum_{k=1}^K \left[a_k\,\cos(2\pi\,\nu_k\,t_n) + b_k\,\sin(2\pi\,\nu_k\,t_n)\right]
  \\
  \nu_k &= \nu_\peak + \left(k - \frac{K - 1}{2}\right)\,\Delta\nu
  \quad \mbox{for $1\leq k\leq K$, $K$ odd,}
\end{align}
where $\nu_\peak$ is the position of one line (preferably a central line)
and $\Delta\nu$ is the asteroseismic ``large'' frequency difference.
In this formulation, $a_0$ and the $a_k,b_k$ comprise a vector of
$2\,K+1$ linear parameters and $\nu_\peak, \Delta\nu$ comprise a
blob of 2 nonlinear parameters.

Under the assumption of Gaussian noise, the log likelihood looks like a
chi-squared:
\begin{align}
  \ln L &= -\frac{1}{2}\,\sum_{n=1}^N \left[\frac{[y_n - \mu_n(t_n)]^2}{\sigma_n^2} + \ln(2\pi\,\sigma_n^2)\right]
  \quad,
\end{align}
where we have included the $\ln\sigma^2$ terms because they might reappear later.

\acknowledgments
It is a pleasure to thank
Andy Casey (Monash),
Stephen Feeney (UCL),
and
Dan Foreman-Mackey (Flatiron)
for valuable work and discussions related to this project.

\software{
\package{Astropy} \citep{astropy, astropy:2018},
\package{IPython} \citep{ipython},
\package{matplotlib} \citep{mpl},
\package{numpy} \citep{numpy}
}

\bibliographystyle{aasjournal}
\bibliography{tess_seismology}

\end{document}


