\documentclass[modern]{aastex63}

% typography
\usepackage[T1]{fontenc}
\setlength{\parindent}{1.\baselineskip}
\newcommand{\acronym}[1]{{\small{#1}}}
\newcommand{\package}[1]{\textsl{#1}}
\newcommand{\changes}[1]{{\textbf{#1}}}

% math
\newcommand{\peak}{\mathrm{peak}}

% aastex parameters
% \received{not yet; THIS IS A DRAFT}
%\revised{not yet}
%\accepted{not yet}
% % Adds "Submitted to " the argument.
% \submitjournal{ApJ}
\shorttitle{}
\shortauthors{bonaca \& hogg}

%@arxiver{}
\usepackage{amsmath}

\begin{document}\sloppy\sloppypar\raggedbottom\frenchspacing % trust me

\title{Likelihood for the forest of asteroseismic modes}

\correspondingauthor{Ana~Bonaca}
\email{ana.bonaca@cfa.harvard.edu}

\author[0000-0002-7846-9787]{Ana~Bonaca}
\affil{Center for Astrophysics | Harvard \& Smithsonian, 60 Garden Street, Cambridge, MA 02138, USA}

\author[0000-0003-2866-9403]{David~W.~Hogg}
\affiliation{Center for Cosmology and Particle Physics, Department of Physics, New York University}
\affiliation{Center for Data Science, New York University}
\affiliation{Max-Planck-Institut f\"ur Astronomie, Heidelberg}
\affiliation{Center for Computational Astrophysics, Flatiron Institute, 162 Fifth Avenue, NY 10010, USA}

\begin{abstract}\noindent % trust me
\end{abstract}

\section{Introduction}
\label{sec:intro}

- successes of asteroseismology

- possible to do bc kepler and tess come close to ideal dataset for asteroseismology: sampling good, timebaseline sufficient to resolve modes
%- no data exactly that, but kepler and tess come close, therefore resolving modes
- typical workflow: fft + peak bagging

- dream: do large-scale asteroseismic missions
- while asteroseismic missions getting larger themselves, e.g., tess, plato, a wealth of time domain data at different cadence
- some work done (cite atlas/asassn paper), but much more possible if able to deal w gaia / lsst cadence
-> need generative model

- sth about why generative model hard / not done so far in asteroseismology (physical model hard / don't have a good theory of amplitudes? so doing fft bc can get frequencies alone easily)
- examples of contexts where generative model worked, why it might be a good idea to try here
- benefits of being able to do inference:
-- can use even suboptimally sampled data, just will get noisier constraints
-- test properties of stars

- luminous red giants valuable for mapping the galaxy
- high amplitude, coherent models -- good asteroseismic targets too, especially for less precise photometry
- model treating modes as coherent in data stream, but they are not in the tess window for sun-like stars, geared towards giants where coherence times are long (would need to use a proper gp or cut data into pieces for sunlike stars)
- not solving all problems for all people, but can do this specific regime
- can take more data than before, maybe map the whole galaxy
- be modest on what we solve
- part of a big project that astronomy should solve

- paper plan:
2) develop likelihood
3) test on aguirre sample: identify where works well and where not
4) blind search: here they are + hopefully make sense
5) discussion


figures:
sec 2: pedagogical -- labeled powerspectrum w nupeak, dnu, bell params 
sec 2: row 1: schematic of different models (just the comb, comb + nupeak marginalization, comb + bell); row 2: likelihood surfaces of the same lightcurve under these models
sec 3: compare likelihood + powerspectra on the bottom for different regimes
sec 4: HRD color-coded by inferre dnu, numax, etc


- beginning of the methods section: two likelihoods we could use
1) comprehensive to include all things we know about stars
2) quick, can search computationally, but doesn't have all the inputs (e.g., no odd/even l, or taking care of them in a weird way to avoid having a parameter, filtering, hard to apply to unevenly sampled streams, like gaia -- new kind of data might require new kinds of hacks if we want something computationally fast)


discussion:
-- talk about all kinds of data regimes: tess (very good data, short timebaseline, slightly gappy), ground-based (very good sampling, rv spec | sampling horrible, like lsst), space-based (precision great, sampling bad, gaia) -- regimes where fft troublesome (nobody has a plan for gaia, even thinks its possible)
-- regimes of timescales (min dt between observations, max T, baseline)
-- in asteroseismic spectrum relevant timescales: 1/numax, 1/dnu, coherence time (1/true width of the modes) -- very separated, so a lot of datasets are in between (giants probably not even resolved in kepler, coherence time longer than 4 yrs -- spinoff undergrad paper for kepler: can we measure coherence time for giant modes)


\section{Likelihood}
\label{sec:lhood}

There are many ways to construct a likelihood function for an
asteroseismic-like frequency comb that generates a light curve.
We will build up in stages of increasing sophistication.
The simplest idea is that the light curve is generated by a set of
$K$ equally spaced frequencies.

Presume that there are $N$ data points (normalized flux measurements)
$y_n$, normalized to have a mean near unity.
The data points are taken at (precisely known) times $t_n$.
Each of these flux measurements has some kind of uncertainty estimate
$\sigma_n$ associated with it and inverse variance $\sigma_n^{-2}$.
The simplest $K$-frequency model is that these flux measurements are
generated by a set of $K$ frequencies $\nu_k$ (where $K$ is odd) as follows:
\begin{align}
  y_n &= \mu_n(t_n) + \mbox{noise}
  \\
  \mu_n(t_n) &= a_0 + \sum_{k=1}^K \left[a_k\,\cos(2\pi\,\nu_k\,t_n) + b_k\,\sin(2\pi\,\nu_k\,t_n)\right]
  \\
  \nu_k &= \nu_\peak + \left(k - \frac{K - 1}{2}\right)\,\Delta\nu
  \quad \mbox{for $1\leq k\leq K$, $K$ odd,}
\end{align}
where $\nu_\peak$ is the position of one line (preferably a central line)
and $\Delta\nu$ is the asteroseismic ``large'' frequency difference.
In this formulation, $a_0$ and the $a_k,b_k$ comprise a vector of
$2\,K+1$ linear parameters and $\nu_\peak, \Delta\nu$ comprise a
blob of 2 nonlinear parameters.

Under the assumption of Gaussian noise, the log likelihood looks like a
chi-squared:
\begin{align}
  \ln L &= -\frac{1}{2}\,\sum_{n=1}^N \left[\frac{[y_n - \mu_n(t_n)]^2}{\sigma_n^2} + \ln(2\pi\,\sigma_n^2)\right]
  \quad,
\end{align}
where we have included the $\ln\sigma^2$ terms because they might reappear later.

\acknowledgments
It is a pleasure to thank
Andy Casey (Monash),
Stephen Feeney (UCL),
and
Dan Foreman-Mackey (Flatiron)
for valuable work and discussions related to this project.

\software{
\package{Astropy} \citep{astropy, astropy:2018},
\package{IPython} \citep{ipython},
\package{matplotlib} \citep{mpl},
\package{numpy} \citep{numpy}
}

\bibliographystyle{aasjournal}
\bibliography{tess_seismology}

\end{document}


